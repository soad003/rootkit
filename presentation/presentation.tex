\documentclass[]{beamer}
\usepackage{etex}
\usetheme[pageofpages=of,% String used between the current page and the
                         % total page count.
          bullet=circle,% Use circles instead of squares for bullets.
          titleline=true,% Show a line below the frame title.
          alternativetitlepage=true,% Use the fancy title page.
          ]{Torino}
\usepackage{hyperref}
\usepackage{listings}


\newcommand{\shellcmd}[1]{\\\indent\indent\texttt{\footnotesize\ #1}\\}
\newcommand{\shellcmdinline}[1]{\texttt{\footnotesize #1}}
\definecolor{darkgreen}{rgb}{0.0,0.5,0.0}
\definecolor{darkred}{rgb}{0.65,0.06,0.06}
\definecolor{mygrey}{rgb}{0.24,0.21,0.2}

\setbeamercolor{structure}{fg=darkgreen}

\setbeamertemplate{footline}
{
  \leavevmode%
  \hbox{%
  \begin{beamercolorbox}[wd=.4\paperwidth,ht=2.25ex,dp=1ex,center]{author in head/foot}%
    \usebeamerfont{author in head/foot}\insertshortauthor
  \end{beamercolorbox}%
  \begin{beamercolorbox}[wd=.6\paperwidth,ht=2.25ex,dp=1ex,center]{title in head/foot}%
    \usebeamerfont{title in head/foot}\insertshorttitle\hspace*{3em}
    \insertframenumber{} / \inserttotalframenumber\hspace*{1ex}
  \end{beamercolorbox}}%
  \vskip0pt%
}


\definecolor{mygreen}{rgb}{0,0.6,0}
\definecolor{mygray}{rgb}{0.5,0.5,0.5}
\definecolor{mymauve}{rgb}{0.58,0,0.82}

\lstset{ %
  backgroundcolor=\color{white},   % choose the background color
  basicstyle=\footnotesize,        % size of fonts used for the code
  breaklines=true,                 % automatic line breaking only at whitespace
  captionpos=b,                    % sets the caption-position to bottom
  commentstyle=\color{mygreen},    % comment style
  escapeinside={\%*}{*)},          % if you want to add LaTeX within your code
  keywordstyle=\color{blue},       % keyword style
  stringstyle=\color{mymauve},      % string literal style
  language=C
}

\author{Clemens Brunner\\Michael Fr\"owis}
\institute{clemens.brunner@student.uibk.ac.at \\ michael.froewis@student.uibk.ac.at}
\title{Rootkit}
\subtitle{Rootkit: Keylogger/Backdoor}
\date{\today}




\begin{document}
\begin{frame}[t,plain]
\titlepage
\end{frame}

\begin{frame}[t]{Outline}
  \tableofcontents
\end{frame}

\section{Introduction}
\begin{frame}[t]{Outline}
  \tableofcontents[currentsection]
\end{frame}


\begin{frame}[t]{Introduction}
  \begin{itemize}
    \item General: Rootkit
    \begin{itemize}
    \item Software
    \item Root privilegs
    \item Masking existence
    \end{itemize}
    \item Our tool: Linux kernel rootkit
    \begin{itemize}
      \item Keylogging
      \item Backdoor
    \end{itemize}
  \end{itemize}
\end{frame}

\subsection{Kernel Modules}

\begin{frame}[t]{Introduction: Kernel Modules}
  \begin{itemize}
    \item Kernel Modules
    \begin{itemize}
      \item No rebuild
      \item No reboot
    \end{itemize}
    \item Example:

    \end{itemize}

 \lstinputlisting{exampleKernel.c}


\end{frame}


\section{Implementation}
\begin{frame}[t]{Outline}
\tableofcontents[currentsection]
\end{frame}


\subsection{Hiding}

\begin{frame}[t]{Hiding}
  \begin{itemize}
    \item Mask existence
    \begin{itemize}
      \item Kernel moduls not visible as a process
    \end{itemize}
    \item \shellcmdinline{lsmod}
    \begin{itemize}
      \item special exported symbol \texttt{extern struct module \_\_this\_module;}
      \item \lstinline{list_del(&(__this_module.list));}
    \end{itemize}
  \end{itemize}
  \lstinputlisting{module.c}
\end{frame}


\subsection{Backdoor}

\begin{frame}[t]{Backdoor}
  \begin{itemize}
    \item Spaning userland process
    \begin{itemize}
      \item \shellcmdinline{netcat -l -p 6666 -e /bin/sh}
      \item \shellcmdinline{call\_usermodehelper}
    \end{itemize}
  \end{itemize}
  \lstinputlisting{backdoor.c}
\end{frame}


\subsection{Keylogging}

\begin{frame}[t]{Keylogging}
  \begin{itemize}
    \item Linux Kernel provides function
    \begin{itemize}
      \item Register a \lstinline{struct notifier_block keyboard_notifier}
    \end{itemize}
    \item \lstinline{keyboard_hook} gets keycode as input
    \begin{itemize}
      \item Mapping: Keycode $\rightarrow$ Character (US)
    \end{itemize}
  \end{itemize}
  \lstinputlisting{keylogging.c}
\end{frame}



\subsection{Networking and Activation}

\begin{frame}[t]{Networking and Activation}
  \begin{itemize}
    \item Magic Packages
    \begin{itemize}
      \item Ping Request where \lstinline{ID == Code}
      \begin{itemize}
        \item[122:] \lstinline{KEYLOGGER_ACTIVATION_CODE}
        \item[123:] \lstinline{KEYLOGGER_DEACTIVATION_CODE}
        \item[124:] \lstinline{HIDEMODULE_ACTIVATION_CODE}
        \item[125:] \lstinline{HIDEMODULE_DEACTIVATION_CODE}
        \item[126:] \lstinline{BACKDOOR_ACTIVATION_CODE}
      \end{itemize}
    \end{itemize}
    \item Backdoor communication
    \begin{itemize}
      \item UDP datagram socket
    \end{itemize}
    \item Sending key charaters
    \begin{itemize}
      \item UDP datagram socket
    \end{itemize}
  \end{itemize}
\end{frame}


\section{Livedemo}
\begin{frame}[t]{Outline}
  \tableofcontents[currentsection]
\end{frame}

\begin{frame}[t]{Livedemo}
   \begin{center} \LARGE{Demo...} \end{center}
\end{frame}

\section{Summary}
\begin{frame}[t]{Outline}
\tableofcontents[currentsection]
\end{frame}

\begin{frame}[t]{Summary}
\begin{itemize}
  \item Simple rootkit: easy
  \item Perfekt rootkit: very hard
\end{itemize}
\end{frame}

\end{document}

